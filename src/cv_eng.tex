%%%%%%%%%%%%%%%%%%%%%%%%%%%%%%%%%%%%%%%
% Deedy CV/Resume
% XeLaTeX Template
% Version 1.0 (5/5/2014)
%
% This template has been downloaded from:
% http://www.LaTeXTemplates.com
%
% Original author:
% Debarghya Das (http://www.debarghyadas.com)
% With extensive modifications by:
% Vel (vel@latextemplates.com)
%
% License:
% CC BY-NC-SA 3.0 (http://creativecommons.org/licenses/by-nc-sa/3.0/)
%
% Important notes:
% This template needs to be compiled with XeLaTeX.
%
%%%%%%%%%%%%%%%%%%%%%%%%%%%%%%%%%%%%%%

\documentclass[letterpaper]{deedy-resume} % Use US Letter paper, change to a4paper for A4

\begin{document}

%----------------------------------------------------------------------------------------
%	TITLE SECTION
%----------------------------------------------------------------------------------------

% \lastupdated % Print the Last Updated text at the top right

\namesection{Evo}{Annus}{ % Your name
% \urlstyle{same}\url{https://www.linkedin.com/in/tavo-annus-4a5631171/} \\ % Your website, LinkedIn profile or other web address
16.10.2001 | Viimsi, Harjumaa \\
\href{mailto:evo.annus@gmail.com}{evo.annus@gmail.com} | +372 5692 6727 % Your contact information
}

%----------------------------------------------------------------------------------------
%	LEFT COLUMN
%----------------------------------------------------------------------------------------

\begin{minipage}[t]{0.33\textwidth} % The left column takes up 33% of the text width of the page

%------------------------------------------------
% Education
%------------------------------------------------

\section{Education}

\subsection{TalTech}

\descript{BSc in Product development and Robotics}
\location{2020 - 2023 | III year}
GPA: 5.0 \\
Done 156 ECTS


\sectionspace % Some whitespace after the section

\subsection{Tallinn Secondary \\School of Science}

\location{Natural Sciences, Programming}
\location{2017 - 2020}
Silver medal

\sectionspace % Some whitespace after the section

\subsection{Viimsi Secondary School}
\location{2008 - 2017}

\sectionspace % Some whitespace after the section

%------------------------------------------------
% Links
%------------------------------------------------

\section{Links}

\href{https://susiik.github.io/}{\bf Portfolio} \\
\href{https://github.com/susIik}{\bf Github} \\

\sectionspace % Some whitespace after the section

%------------------------------------------------
% Skills
%------------------------------------------------

\section{Skills}

\subsection{Languages}

Estonian - Native Speaker \\
English - C1 \\
Russian - Basic communication \\

\sectionspace % Some whitespace after the section

%------------------------------------------------

\subsection{Drivers licenses}

B - category \\

\sectionspace % Some whitespace after the section

\subsection{Engineering Skills}

\sectionspace

\descript{Mechanics}
\location{CAD / CAM}
Solidworks \textbullet{} Siemens NX \textbullet{} Solid Edge \\
\location{Welding}
MIG/MAG \textbullet{} Shielded metal arc welding \\

\sectionspace

\descript{Electronics}
\location{Soldering}
\location{Arduino}
\location{STM32 nucleo}

\sectionspace

\descript{Programming}
\location{Python}
\location{C}
\location{Matlab}


\sectionspace % Some whitespace after the section

%----------------------------------------------------------------------------------------

\end{minipage} % The end of the left column
\hfill
%
%----------------------------------------------------------------------------------------
%	RIGHT COLUMN
%----------------------------------------------------------------------------------------
%
\begin{minipage}[t]{0.66\textwidth} % The right column takes up 66% of the text width of the page

%------------------------------------------------
% Experience
%------------------------------------------------

\section{Experience}

\runsubsection{Neptune First}
\descript{| Mechanical Engineer}

\location{April 2022 - Present}
\vspace{\topsep} % Hacky fix for awkward extra vertical space
\begin{tightitemize}
  \item We are developing a device Trimemory, that makes possible exact sail curvature measurements and therefore it's possible to optimise the sail shape.
  \item I am designing parts with \textbf{Solidworks} and then make these parts with \textbf{3D printing}.
  \item I am optimising \textbf{the production process} for that device.
  \item I am changing the design of the device to minimize production cost and to increase durability of the device.
\end{tightitemize}

\sectionspace % Some whitespace after the section

%------------------------------------------------

\runsubsection{Milrem Robotics}
\descript{| Mechanical Engineer}

\location{Juuli 2022}
\vspace{\topsep} % Hacky fix for awkward extra vertical space
\begin{tightitemize}
  \item Projekteerisin \textbf{Solidworks} tarkvara abil THeMIS platvormile ühilduvat Thedered Follow-Me juhtimissüsteemi.
  \item Prototüübi jaoks vajalikud detailid valmistasin \textbf{3D printimise} teel.
  \item Valisin vajalikud ostutooted, et vähendada eridetailide valmistamise vajadust.
  \item Monteerisin kokku lõpliku toote ja paigaldasin selle THeMISele.
  \item \textbf{Testisin} koos teiste projekti tiimi liikmetega valminud prototüüpi ja muutsin disaini vastavalt vajadusele.
\end{tightitemize}

\sectionspace % Some whitespace after the section

%------------------------------------------------

\runsubsection{Kitman Thulema}
\descript{| Projekteerija}

\location{Juuni 2022}
\vspace{\topsep} % Hacky fix for awkward extra vertical space
\begin{tightitemize}
  \item Disainerite jooniste alusel koostasin \textbf{Solid Edge} tarkvara abil tootmisesse minevate \textbf{lehtmetallist} ja \textbf{puidust} toodete mudelid ja joonised.
  \item Valisin \textbf{materjale} ja \textbf{tootmisprotsesse} lähtuvalt kliendi nõuetest tootele.
  \item Vastutasin \textbf{3D printeri} töökorras olemise ja sellega detailide printimise eest.
\end{tightitemize}

\sectionspace % Some whitespace after the section

%------------------------------------------------

%------------------------------------------------
% Projects
%------------------------------------------------

\section{Projects}

%\runsubsection{3D bike model}
%\descript{| School Project}
%
%\location{2020}
%\vspace{\topsep} % Hacky fix for awkward extra vertical space
%\begin{tightitemize}
%  \item As a group project we designed and modeled bicycle using \textbf{Solidworks CAD} software.
%  \item I personally modeled derailleur, break, saddle and chain. I also modeled some less significant details.
%  \item Our bicycle was the second-best project that year.
%\end{tightitemize}
%
%\sectionspace % Some whitespace after the section

%------------------------------------------------

\runsubsection{Electrical Skateboard}
\descript{| Personal Project}

\location{2021 - 2022}
\begin{tightitemize}
  \item I began working on this project because I wanted to make a skateboard that doesn't require a remote to control its speed.
  \item Speed controlling is made possible by using \textbf{strain gauge} sensors, that are mounted on the trucks.
  \item \textbf{Arduino} is used to process the data coming from sensors and to output the required PWM signal for motor speed controlling.
  \item During this project I learned \textbf{soldering}, \textbf{motor speed controlling}, using \textbf{strain gauges} and \textbf{Arduino} programming.
\end{tightitemize}

\sectionspace % Some whitespace after the section

%------------------------------------------------

\runsubsection{Autonomous boat}
\descript{| Robotics Club Project}

\location{2022}
\begin{tightitemize}
  \item As a group we designed and built a boat, that has to complete a lap on the track as fast as possible.
  \item Boat hull is modeled in \textbf{Solidworks} and \textbf{3D printed}.
  \item Electronics is controlled by \textbf{STM32 nucleo f303k8}, which is programmed in \textbf{C language}.
  \item Controller gets the data from \textbf{IR sensors}, that measure the distance from an object. An optimal driving path can be calculated using this data.
\end{tightitemize}

% \begin{tightitemize}
%   \item \textbf{\href{https://github.com/kilpkonn/neowatch}{neowatch}} - Creator of a Rust alternative to \textit{watch} command with addons.
%   \item \textbf{\href{https://github.com/DEVELOPEST}{gtm}} - One of the creators of an open source time tracking app \textit(similar to wakatime).
%   \item \textbf{\href{https://github.com/kilpkonn/SportsApp}{SportsApp}} - Creator of an open source Android app for tracking outdoor trainings.
%   \item \textbf{\href{https://github.com/kilpkonn/LonePlayer}{LonePlayer}} - One of two creators of 2D platformer game.
%   \item \textbf{\href{https://github.com/kilpkonn/GomokuGUI}{GomokuGUI}} - Creator of an open source Gomoku game made for AI \textit{(TalTech uses it for Java course assingment)}.
%   \item \textbf{\href{https://veloren.net/}{veloren}} - Contributor, improved camera clipping, misc other minor improvements.
% \end{tightitemize}

\sectionspace % Some whitespace after the section

%------------------------------------------------
% Research
%------------------------------------------------

% \section{Research}

% \runsubsection{Cornell Robot Learning Lab}
% \descript{| Head Undergrad Research}
%
% \location{Jan 2014 – Present | Ithaca, NY}
% Worked with \textbf{\href{http://www.cs.cornell.edu/~ashesh/}{Ashesh Jain}} and \textbf{\href{http://www.cs.cornell.edu/~asaxena/}{Prof Ashutosh Saxena}} to create \textbf{PlanIt}, a tool which learns from large scale user preference feedback to plan robot trajectories in human environments. Publication submitted.
%
% \sectionspace % Some whitespace after the section

%------------------------------------------------

% \runsubsection{Cornell Phonetics Lab}
% \descript{| Head Undergraduate Researcher}
%
% \location{Mar 2012 – May 2013 | Ithaca, NY}
% Lead the development of \textbf{QuickTongue}, the first ever breakthrough tongue-controlled game with \textbf{\href{http://conf.ling.cornell.edu/~tilsen/}{Prof Sam Tilsen}} to aid in Linguistics research. Publication submitted.
%
% \sectionspace % Some whitespace after the section

%------------------------------------------------
% Awards
%------------------------------------------------

%\section{Awards}
%
%\begin{tabular}{rll}
%2021	 & 1\textsuperscript{st}/20 & Robocode - Free for all \\
%2020	 & 3\textsuperscript{rd}/15 & Robocode - Main tournament \\
%2020     & 3\textsuperscript{rd}/30 & NCPC Qualification (Estonia) \\
%2019     & 1\textsuperscript{st}/25 & Robocode - Main tournament \\
%2018 & 8\textsuperscript{th} & Physics Olympiad Natianal contest \\
%2016 & 1\textsuperscript{st} & Physics Open (Natianal contest) \\
%\end{tabular}

%\sectionspace % Some whitespace after the section

%------------------------------------------------
% Hobbies
%------------------------------------------------

\section{Hobbies}

Sailing - Competing for National Team \\
Investing, Reading\\

\sectionspace % Some whitespace after the section

%----------------------------------------------------------------------------------------

\end{minipage} % The end of the right column

%----------------------------------------------------------------------------------------
%	SECOND PAGE (EXAMPLE)
%----------------------------------------------------------------------------------------

%\newpage % Start a new page

%\begin{minipage}[t]{0.33\textwidth} % The left column takes up 33% of the text width of the page

%\section{Example Section}

%\end{minipage} % The end of the left column
%\hfill
%\begin{minipage}[t]{0.66\textwidth} % The right column takes up 66% of the text width of the page

%\section{Example Section 2}

%\end{minipage} % The end of the right column

%----------------------------------------------------------------------------------------

\end{document}
